\documentclass[a4paper]{article}
\usepackage{lmodern}

\usepackage{mathtools}

\usepackage{array}
\usepackage{booktabs}
\usepackage{multicol}

\usepackage[margin=1cm]{geometry}
\usepackage{diffcoeff}
\diffdef { p }
{
    op-symbol = \partial ,
    left-delim = \left .,
    right-delim = \right | ,
    subscr-nudge = 0 mu
}

\usepackage{mleftright}
\mleftright

\usepackage{pdflscape}
\usepackage{sagetex}

% https://tex.stackexchange.com/questions/281862/drawing-a-double-line
\newcommand{\doublerule}[1][.4pt]{%
  \noindent
  \makebox[0pt][l]{\rule[.7ex]{\linewidth}{#1}}%
  \rule[.3ex]{\linewidth}{#1}}

\usepackage{xcolor}

\definecolor{imperialorange}{RGB}{210,64,0}
\definecolor{imperialbrick}{RGB}{165,25,0}

\newenvironment{MyColorPar}{%
        \leavevmode\color{imperialbrick}\ignorespaces%
    }{%
}%

\usepackage{cancel}
\usepackage{cleveref}

\begin{document}

\begin{landscape}
    \large
    \centering
    \begin{minipage}{0.480\columnwidth}
        \begin{alignat}{2}
            \diffp{u}{t} - \triangle u & = f  &\quad& \text{ in } \Omega \\
            u & = g  &\quad& \text{ on } \partial\Omega 
        \end{alignat}
    \end{minipage}
    \hfill
    \begin{minipage}{0.480\columnwidth}
        \begin{alignat}{2}
            \diffp{u}{t} - \nabla \cdot \left(D\left(u\right) \nabla u\right) & = 0  &\quad& \text{ in } \Omega  \\
            D(u) \diffp{u}{x}{0} = -\beta; & \quad   D(u) \diffp{u}{x}{1} = 0 
        \end{alignat}
    \end{minipage}

    \bigskip
    \noindent\rule{\columnwidth}{1pt}

    \begin{minipage}{0.4800\columnwidth}
        \begin{equation}
            \frac{u^n - u^{n-1}}{k_n} - \triangle \left(\left(1-\theta\right)u^{n-1} + \theta u^n\right) \\ = \left(1-\theta\right)f\left(x,t_{n-1}\right) + \theta f\left(x,t_n\right)
        \end{equation}
    \end{minipage}
    \hfill
    \begin{minipage}{0.4800\columnwidth}
        \begin{alignat}{2}
            \begin{multlined}
                \frac{u^n - u^{n-1}}{k_n} - D\left(u^\ast\right) \left(\left(1-\theta\right)u^{n-1} + \theta\, u^n\right) = 0, \quad  \\
                \qquad\qquad D\left(u^\ast\right) = f(u^{n-k}), \ k = 1,2,\dots m
            \end{multlined} 
        \end{alignat}
    \end{minipage}

    \bigskip
    \noindent\rule{\columnwidth}{1pt}

    \begin{flushleft}
        \underline{Spatial Discretisation}

        \medskip
        $u^n(x) \approx u^n_h(x) \coloneqq \sum_j U^n_j \phi_j(x)$
    \end{flushleft}

    \begin{minipage}{0.480\columnwidth}
        \begin{equation}
            M\frac{U^n - U^{n-1}}{k_n} + A \left(\left(1-\theta\right)U^{n-1} + \theta\, U^n\right) =  \left( 1-\theta \right) F^{n-1} + \theta F^n 
        \end{equation}
    \end{minipage}
    \hfill
    \begin{minipage}{0.4800\columnwidth}
        \begin{equation}
            M\frac{U^n - U^{n-1}}{k_n} + A D(u^*) \left(\left(1-\theta\right)U^{n-1} + \theta\, U^n\right) =  (w\cdot-\beta){\Huge\mid}_{x=0}
        \end{equation} 
    \end{minipage}

    \bigskip
    \noindent\rule{\columnwidth}{1pt}

    \begin{minipage}{0.48\columnwidth}
        \begin{equation}
            \left(M + k_n \theta A\right) U^n = \left(M - k_n A \left(1-\theta\right)\right) U^{n-1} + k_n\left(\left( 1-\theta \right) F^{n-1} + \theta F^n\right)
        \end{equation}
    \end{minipage}
    \hfill
    \begin{minipage}{0.515\columnwidth}
        \begin{equation}
            \left(M + k_n \theta D\left(u^\ast\right) A \right) U^n = \left(M - k_n D\left(u^\ast\right) A \left(1-\theta\right)\right) U^{n-1} + k_n(w\cdot-\beta){\Huge\mid}_{x=0}
        \end{equation} 
        \begin{equation}
            \left(M + k_n \theta D\left(u^\ast\right) A \right) U^n = \left(M - k_n D\left(u^\ast\right) A \left(1-\theta\right)\right) U^{n-1} + \,  {-k_n}\beta\ w\left(0\right)
        \end{equation}
    \end{minipage}
\end{landscape}

    \newpage
    \setcounter{equation}{0}
    With $\mathbf{x} = (x,y)$,
    \begin{alignat}{3}
        \diffp{u(\mathbf{x},t)}{t} &= \nabla\!\cdot\!D(\mathbf{x})\, \nabla u(\mathbf{x},t) - \Sigma_a(\mathbf{x})\!\cdot\!u(\mathbf{x},t) + S(\mathbf{x},t)   \quad && \text{ in } \Omega: (0,b) \times (0,b) \label{eq:full_diffusion_eq} \\
        \intertext{Using the vector calculus identity,}
        \nabla \cdot \big(D(x) \, \nabla\mathbf{u}\big)  &=  D(x) \,\big(\nabla\!\cdot \!\nabla\mathbf{u}\big) +  \nabla\mathbf{u} \cdot \big(\nabla D(x)\big) \label{eq:vec_calc_id} \\
        \intertext{If diffusion coefficient is spatially independent, i.e.\ D(x) = D, \cref{eq:vec_calc_id} becomes}
        \nabla \cdot \big(D \, \nabla u\big)  &=  D \,\big(\nabla\!\cdot \!\nabla u\big) +  \nabla u \cdot \big(\cancelto{0}{\nabla D}\big)  \\
        \nabla \cdot \big(D \, \nabla u\big)  &=  D \,\triangle u \label{eq:D_const_with_laplacian} \\
        \intertext{Applying \cref{eq:D_const_with_laplacian} to \cref{eq:full_diffusion_eq}, and additionally with constant absorption cross-section $\Sigma_a(x) = \Sigma_a$,}
        \diffp{u(\mathbf{x},t)}{t} &= D\, \triangle u(\mathbf{x},t) - \Sigma_a\!\cdot\!u(\mathbf{x},t) + S(\mathbf{x},t) \quad && \text{ in } \Omega: (0,b) \times (0,b) \label{eq:strong_form_pde} \\
        u(\mathbf{x},t)_{\Huge{\lvert}\scriptsize\substack{x=0\\\forall y}}   &= 0\,;\quad u(\mathbf{x},t)_{\Huge{\lvert}\scriptsize\substack{x=b\\\forall y}} = 0  \label{eq:bc1_2d}\\[1ex]
        D \, \nabla u(\mathbf{x},t)_{\substack{y=0\\\forall x}} \cdot \mathbf{n}&= 0\,; \quad D \, \nabla u(\mathbf{x},t)_{\substack{y=b\\\forall x}} \cdot \mathbf{n}= 0\label{eq:bc2_2d}
    \end{alignat}

    % \doublerule

    \begin{MyColorPar}
        The only change is to move to 1D, $\mathbf{x} = x,\, \Omega:x \in (0,b)$ whilst retaining a ``similar'' set of boundary conditions
        \begin{align}
            u(\mathbf{x},t)_{\Huge{\lvert}\scriptsize\substack{x=0}}   &= 0 \label{eq:dirichelet_bc1_1d}\\[1ex]
             D \nabla u(\mathbf{x},t)_{x=b}\cdot\mathbf{n} &= 0\label{eq:neumann_bc2_1d}
        \end{align}
    \end{MyColorPar}

    % Assume $u(\mathbf{x},t) = A \sin\left(\omega t\right)\big(b\, \mathbf{x} - \mathbf{x}^2\big)$. 
    \begin{sagesilent}
        reset()
        var('x','t','omega','A','b','Sigma_a','D','u')
        u = A*sin(omega*t)*(x^3 - 2*b*x^2 + b^2*x)
    \end{sagesilent}
    If we apply a Homogenous Neumann BC at the RHS, we need to have an analytical solution with a suitable form that facilities this.
    \begin{equation}
        u(x,t) = \sage{u} 
    \end{equation}
    % $\diffp{u}{x} = \sage{diff(u,x)}$
    % \sage{(diff(phi,t) + (Sigma_a*phi - D*diff(phi,x,2)).collect(sin(omega*t))).collect_common_factors()}

    When using the Method of Manufactured Solutions (MMS) with the above assumed $u(\mathbf{x},t)$, the source term is
    \begin{equation}
        % S = A \Big ( \omega \cos(\omega t) \left(b \, \mathbf{x} - \mathbf{x}^2\right) + \sin( \omega t ) \big(\Sigma_a  ( b \, \mathbf{x} - \mathbf{x}^2) + 2D \big) \Big )
        S(x,t) = \sage{(diff(u,t) + (Sigma_a*u - D*diff(u,x,2)).collect_common_factors()).collect_common_factors()}
    \end{equation}

    % Multiply \cref{eq:D_const_with_laplacian} from the left by $\psi$ and
    \medskip
    \underline{Discretisation}
    \medskip

    Now, consider the standard vector calculus identity where $\psi$ is a scalar function
    \begin{align}
        \nabla \cdot (\psi \mathbf{A}) &= \psi \nabla \cdot \mathbf{A} + \mathbf{A} \cdot \nabla \psi \\
        \intertext{Using $\mathbf{A} = \nabla u$}
        \nabla \cdot (\psi \nabla u) &= \psi \nabla \cdot \nabla u + \nabla u \cdot \nabla \psi \\
        \intertext{Including the scalar value D in the above equation, we get,}
        \nabla \cdot (\psi D \nabla u) &= \psi \nabla \cdot D \nabla u + \nabla u \cdot D \nabla \psi \\
        \intertext{Integrating the above over the domain $\Omega$,}
        \int_\Omega \nabla \cdot (\psi D \nabla u)\, \mathrm{d}\Omega &= \int_\Omega \psi \nabla \cdot D \nabla u\, \mathrm{d}\Omega + \int_\Omega\nabla u \cdot D \nabla \psi \, \mathrm{d}\Omega\\
        \oint_{\partial\Omega} \psi D  \nabla u\cdot \mathbf{n}\, \mathrm{d}\Omega &= \int_\Omega \psi \nabla \cdot D \nabla u\, \mathrm{d}\Omega + \int_\Omega\nabla u \cdot D \nabla \psi \, \mathrm{d}\Omega\\
        \intertext{Rearranging terms of the above, we get}
        \int_\Omega \psi \nabla \cdot D \nabla u\, \mathrm{d}\Omega & = - \int_\Omega\nabla u \cdot D \nabla \psi \, \mathrm{d}\Omega + \oint_{\partial\Omega} \psi D  \nabla u\cdot \mathbf{n}\, \mathrm{d}\Omega \\
    \intertext{Using the bilinear notation for the domain and boundary integrals for the RHS},
        \int_\Omega \psi \nabla \cdot D \nabla u\, \mathrm{d}\Omega & = - \big(D \nabla u,  \nabla \psi\big)_\Omega  +  \big(\psi \,, D \,  \nabla u\cdot \mathbf{n}\big)_{\partial\Omega} \\
        \int_\Omega \psi \nabla \cdot D \nabla u\, \mathrm{d}\Omega & = - \big(D   \nabla \psi, \nabla u \big)_\Omega  +  \big(\psi \,, D \,  \nabla u\cdot \mathbf{n}\big)_{\partial\Omega} \\
        \intertext{Substituting discretised form of test function $\psi = \psi_i$ \& $u_h = \Sigma_j U_j(t)\psi_j$ \& where the bilinear notation () denotes summations},
        \int_\Omega \psi \nabla \cdot D \nabla u\, \mathrm{d}\Omega & = - \big(D U(t) \nabla \psi_i, \nabla \psi_j \big)_\Omega  +  \big(\psi_i \,, D \, \nabla u\cdot \mathbf{n}\big)_{\partial\Omega} \label{eq:weak_form_diffusion_laplacian}
    \end{align}
    \begin{multline}
        \int_\Omega \psi \nabla \cdot D \nabla u\, \mathrm{d}\Omega = - \big(D U(t) \nabla \psi_i, \nabla \psi_j \big)_\Omega  +  \big(\psi_i \,, D \, \nabla u\cdot \mathbf{n}\big)_{\substack{x=0\\\forall y}} +  \big(\psi_i \,, D \, \nabla u\cdot \mathbf{n}\big)_{\substack{x=b\\\forall y}} \\ +  \big(\psi_i \,, D \, \nabla u\cdot \mathbf{n}\big)_{\substack{y=0\\\forall x}} +  \big(\psi_i \,, D \, \nabla u\cdot \mathbf{n}\big)_{\substack{y=b\\\forall x}}
    \end{multline}
    With homogenous dirichelet BCs at $x=0$ and $x=b$, and with homogenous Neumann Conditions at $y=a$ and $y=b$,
    \begin{multline}
        \int_\Omega \psi \nabla \cdot D \nabla u\, \mathrm{d}\Omega = - \big(D U(t) \nabla \psi_i, \nabla \psi_j \big)_\Omega  +  \big(\cancelto{0}{\psi_i} \,, D \, \nabla u\cdot \mathbf{n}\big)_{\substack{x=0\\\forall y}} +  \big(\cancelto{0}{\psi_i} \,, D \, \nabla u\cdot \mathbf{n}\big)_{\substack{x=b\\\forall y}} \\ +  \big(\psi_i \,, \cancelto{0}{D \, \nabla u\cdot \mathbf{n}}\big)_{\substack{y=0\\\forall x}} +  \big(\psi_i \,,\cancelto{0}{D \, \nabla u\cdot \mathbf{n}}\big)_{\substack{y=b\\\forall x}}
    \end{multline}
    \begin{align}
        \int_\Omega \psi \nabla \cdot D \nabla u\, \mathrm{d}\Omega &= - \big(D U(t) \nabla \psi_i, \nabla \psi_j \big)_\Omega \label{eq:weak_form_diffusion_laplacian_2d_after_homogenous_bcs}\\
        \int_\Omega \psi \nabla \cdot D \nabla u\, \mathrm{d}\Omega &= - \big( D \nabla \psi_i, \nabla \psi_j \big)_\Omega U(t) \\
        \int_\Omega \psi \nabla \cdot D \nabla u\, \mathrm{d}\Omega &= - \mathcal{D} U(t) 
    \end{align}
    The weak form of the problem \crefrange{eq:strong_form_pde}{eq:bc2_2d} is:

    \begin{equation}
        M \diff{U(t)}{t} = -\mathcal{D} U(t) - \mathcal{A} U(t) + \mathcal{S}(t)
    \end{equation}
    where
    \begin{align}
        M_{ij} &= \big(\psi_i,\psi_j)_\Omega \\
        \mathcal{D}_{ij} &= \big(D \nabla \psi_i,\nabla \psi_j)_\Omega \\
        \mathcal{A}_{ij} &= \big(\Sigma_a \psi_i,\psi_j)_\Omega \\
            \mathcal{S}_{i}(t) &= \big(\psi_i,S(x,t))_\Omega
    \end{align}

    \begin{MyColorPar}
        The only change is to move to 1D. $\mathbf{x} = x,\, \Omega:x \in (0,b)$ whilst retaining a ``similar'' set of boundary conditions. 
        So, splitting the boundary term in \cref{eq:weak_form_diffusion_laplacian},
        \begin{align}
            \int_\Omega \psi \nabla \cdot D \nabla u\, \mathrm{d}\Omega &= - \big(D U(t) \nabla \psi_i, \nabla \psi_j \big)_\Omega  +  \big(\psi_i \,, D \, \nabla u\cdot \mathbf{n}\big)_{x=0} + \big(\psi_i \,, D \, \nabla u\cdot \mathbf{n}\big)_{x=b} \label{eq:weak_form_diffusion_laplacianterm_1d_expanded_boundary_integrals}
            \intertext{Considering the BCs \cref{eq:dirichelet_bc1_1d} and \cref{eq:neumann_bc2_1d}, we get}
            \int_\Omega \psi \nabla \cdot D \nabla u\, \mathrm{d}\Omega &= - \big(D U(t) \nabla \psi_i, \nabla \psi_j \big)_\Omega  +  \big(\cancelto{0}{\psi_i} \,, D \, \nabla u\cdot \mathbf{n}\big)_{x=0} + \big(\psi_i \,, D \, \cancelto{0}{\nabla u\cdot \mathbf{n}}\big)_{x=b} \\
            \int_\Omega \psi \nabla \cdot D \nabla u\, \mathrm{d}\Omega &= - \big(D U(t) \nabla \psi_i, \nabla \psi_j \big)_\Omega
        \end{align}
        which is identical to \cref{eq:weak_form_diffusion_laplacian_2d_after_homogenous_bcs} and hence the rest of the weak form derivation and equations remain the same.
    \end{MyColorPar}

This was tested and works fine.

    % \begin{minipage}{0.51\columnwidth}
    %     \begin{center}
    %         \underline{Original (MMS)}
    %     \end{center}
    %     \begin{align}
    %         \!\!\!\diffp{u(x,t)}{t} &= \nabla\!\cdot\!D(x) \nabla u(x,t) - \Sigma_a(x) u(x,t) + S(x,t)   \text{ in } \Omega: (0,b) \times (0,b) \\
    %         \diffp{u(x,t)}{t} &= \nabla \cdot D\, \nabla u(x,t) - \Sigma_a\, u(x,t) + S(x,t) \text{ in } \Omega: (0,b) \times (0,b) \\
    %         \diffp{u(x,t)}{t} &= D\, \triangle u(x,t) - \Sigma_a\, u(x,t) + S(x,t) \text{ in } \Omega: (0,b) \times (0,b)
    %         % \end{align}
    %         \intertext{Assume $u(x,t) = A \sin\left(\omega t\right)\big(b x - x^2\big)$,}
    %         % \begin{align}
    %         u(0,t) &= 0\,; \ u(b,0) = 0 \\[1ex]
    %         \diffp{u(x,t)}{x}{\mathrlap{x=0}} &= 0\,; \ \diffp{u(x,t)}{x}{\mathrlap{x=b}} = 0
    %     \end{align}
    %         \begin{sagesilent}
    %             var('phi','x','t','omega','A','b','Sigma_a','D')
    %             phi = A*sin(omega*t)*(b*x - x^2)
    %         \end{sagesilent}
    %         \begin{equation}
    %             \sage{(diff(phi,t) + (Sigma_a*phi - D*diff(phi,x,2)).collect(sin(omega*t))).collect_common_factors()}
    %         \end{equation}
    %     \end{minipage}
        % \hfill
        % \begin{minipage}{0.475\columnwidth}
        %     \begin{center}
        %         \underline{Modification from Original (MMS)}
        %     \end{center}
        %     \raggedright
        %     \begin{align}
        %         \!\!\!\diffp{u(x,t)}{t} &= \nabla\!\cdot\!D(x) \nabla u(x,t) + S(x,t)  \text{ in } \Omega: (0,b),\, b = 1\\
        %         \diffp{u(x,t)}{t} &= \nabla \cdot D\, \nabla u(x,t) + S(x,t)   \text{ in } \Omega: (0,b),\, b = 1\\
        %         \diffp{u(x,t)}{t} &= D\, \triangle u(x,t) + S(x,t)  \text{ in } \Omega: (0,b),\, b = 1\\
        %         \intertext{Assume $u(x,t) = A \sin\left(\omega t\right)\big(b x - x^2\big),\, b=1$,}
        %         % \begin{align}
        %         \vphantom{u(0,t) = 0\,; \ u(b,0) = 0,\ b =1} \notag\\[1ex]
        %         \diffp{u(x,t)}{x}{\mathrlap{x=0}} &= -\beta\,; \ \diffp{u(x,t)}{x}{\mathrlap{\substack{x=b,\\b=1}}} = 0
        %     \end{align}
        %     \begin{equation}
        %         \sage{((diff(phi,t) - D*diff(phi,x,2)).collect_common_factors()).subs(b==1)}
        %     \end{equation}
        % \end{minipage}

\end{document}
