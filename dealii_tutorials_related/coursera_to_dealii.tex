\documentclass[a4paper]{article}
\usepackage{lmodern}

\usepackage{mathtools}

\usepackage{array}
\usepackage{booktabs}
\usepackage{multicol}

\usepackage[margin=1cm]{geometry}

\begin{document}
\begin{tabular}{@{} l c c l @{}} \toprule
    % Coursera Nomenclature & Coursera Symbol & Deal II \\
    \multicolumn{2}{c}{Coursera} & \multicolumn{2}{c}{Deal II} \\
    \cmidrule(r){1-2} \cmidrule(l){3-4}
    Nomenclature & Symbol & Symbol & Nomenclature \\
    \midrule
    Weighting function & $w$ & $\varphi$ & Test function \\
    Global finite-dimensional trial solution  & $u^h$ & $u_h$ & \emph{Same meaning} \\
    Global degrees of freedom & $\vec{d}$ & $U$ & \emph{Same meaning} \\
    Finite element basis (shape) functions & $N$ & $\varphi$ & Finite element shape functions  \\
    Parent domain, \emph{bi-unit measure} & $[-1,1]^\text{dim}$ & $[0,1]^\text{dim}$ & Reference cell, \emph{unit measure}  \\
    Element & $e$ & $K$ & Cell  \\
    Element conductivity matrix & $\underline{K}$ & $A^k$ & \emph{No specific name}   \\
    Gauss quadrature index & $l$ & $q$ & \emph{Same meaning}   \\
    \bottomrule
\end{tabular}

\bigskip
For a square mesh, the total number of vertices is $(2^r + 1)^2$, where $r$ is the number of global refinements of the mesh
\end{document}
